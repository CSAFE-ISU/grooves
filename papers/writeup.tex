% !TeX program = pdfLaTeX
\documentclass[12pt]{article}
\usepackage{amsmath}
\usepackage{amsthm}
\usepackage{amssymb}
\usepackage{bbm}
\usepackage{graphicx,psfrag,epsf}
\usepackage{enumerate}
%\usepackage[numbers]{natbib}
\usepackage[nomarkers]{endfloat}
\usepackage{natbib}
\setcitestyle{numbers} 
\makeatletter % Reference list option change
\renewcommand\@biblabel[1]{#1. } % from [1] to 1
\makeatother %

\usepackage{booktabs}
\usepackage{longtable}
\usepackage{array}
\usepackage{adjustbox}
\usepackage{multirow}
\usepackage{subfig}
\usepackage[table,xcdraw]{xcolor}
\usepackage{wrapfig}
\usepackage{float}
%\usepackage{colortbl}
\usepackage[colorlinks]{hyperref}
\hypersetup{
  colorlinks=true,
  citecolor=black,
  linkcolor=black,
  urlcolor=blue}
  
\usepackage{pdflscape}
\usepackage{tabu}
\usepackage{threeparttable}
\usepackage{url} % not crucial - just used below for the URL

\usepackage{etoolbox}% http://ctan.org/pkg/etoolbox
\makeatletter
\patchcmd{\subsection}{\bfseries}{\relax}{}{}% Non-bold \subsection
\patchcmd{\subsubsection}{\bfseries}{\relax}{}{}% Non-bold \subsection
\makeatother


%\pdfminorversion=4
% NOTE: To produce blinded version, replace "0" with "1" below.
\newcommand{\blind}{0}

% DON'T change margins - should be 1 inch all around.
\addtolength{\oddsidemargin}{-.5in}%
\addtolength{\evensidemargin}{-.5in}%
\addtolength{\textwidth}{1in}%
\addtolength{\textheight}{1.3in}%
\addtolength{\topmargin}{-.8in}%

\newenvironment{definition}[1]% environment name 
{% begin code 
  \par\vspace{.75\baselineskip}\noindent 
  \textbf{Definition (#1)}\begin{itshape}% 
  \par\vspace{.5\baselineskip}\noindent\ignorespaces 
}% 
{% end code 
  \end{itshape}\ignorespacesafterend 
}

\providecommand{\tightlist}{%
  \setlength{\itemsep}{0pt}\setlength{\parskip}{0pt}}

\begin{document}

\def\spacingset#1{\renewcommand{\baselinestretch}%
{#1}\small\normalsize} \spacingset{1}



%%%%%%%%%%%%%%%%%%%%%%%%%%%%%%%%%%%%%%%%%%%%%%%%%%%%%%%%%%%%%%%%%%%%%%%%%%%%%%

\if0\blind
{
  \title{\bf Automated Groove Identification in 3D Bullet Land Scans (we'll change
the title)}

  \author{
        Kiegan Rice \thanks{The authors gratefully acknowledge \ldots{}} \\
    Department of Statistics, Iowa State University\\
     and \\     Nathaniel Garton \\
    Department of Statistics, Iowa State University\\
     and \\     Ulrike Genschel \\
    Department of Statistics and CSAFE, Iowa State University\\
     and \\     Heike Hofmann \\
    Department of Statistics and CSAFE, Iowa State University\\
      }
  \maketitle
} \fi

\if1\blind
{
  \bigskip
  \bigskip
  \bigskip
  \begin{center}
    {\LARGE\bf Automated Groove Identification in 3D Bullet Land Scans (we'll change
the title)}
  \end{center}
  \medskip
} \fi

\bigskip
\begin{abstract}

\end{abstract}

\noindent%
{\it Keywords:} 3 to 6 keywords, that do not appear in the title
\vfill

\newpage
\spacingset{1.45} % DON'T change the spacing!

\section{Background}

Thanks to \cite{Hamby}, we can do stuff. \citet{Hamby}

\section{Data Source}

\section{Methodology}

We first need to remove the global structure of the bullet land.

\subsection{Global Structure Removal}

The non-traditional data structure necessitates employing
non-traditional methods to model and remove the global structure. The
data are made up of two competing structures: the LEA data, of which we
would like to model the global structure, and the GEA data, which we
would like to consider as outlying data. Traditional statistical
modeling techniques minimize the least squared vertical distance from
each data point to a fit line; this results in undue influence by GEA
points, which pull any fit lines towards their unusual points.

While bullets are traditionally circular, it is unwise to use a rigidly
quadratic model to fit the global structure. We cannot assume that fired
bullets will retain a neatly circular shape, especially at the level of
detail scans are captured. The significant amount of physical pressure
that acts upon bullets as they are fired through a barrel also can lead
to some warping or slight deformations \emph{(find a citation from JFS
or AFTE about warping/deformation of bullets?)}. Finally, the placement
of the land relative to the plane of reference when a 3D scan is being
captured can vary slightly, meaning that the 2D crosscuts can be
slightly tilted or rotated. This will not translate into a clean
quadratic-shaped(?) crosscut.

To avoid the potential risks arising from using a quadratic linear
model, we instead use a locally weighted regression (LOESS) which fits
linear regression models on small pieces of the data and combines
predictions to result in a non-parametric predicted fit of the data
structure.

However, since LOESS is still rooted in traditional regression
techniques, it is unable to adequately identify and address the
separation between GEA and LEA data structures. To address this, we
implement a robust version of LOESS which iteratively downweights
unusual data points and re-fits a LOESS model to each land. This robust
LOESS is an adapted version of the robust LOESS proposed by
\cite{Cleveland1}.

This model is fit as follows: (add more formulaic language here\ldots{})

\begin{enumerate}
\def\labelenumi{\arabic{enumi}.}
\tightlist
\item
  Fit a LOESS model (span = 1) to an entire crosscut to predict y using
  values of x. Assign weights of 1 to each data point for this fitting
  procedure.\\
\item
  Obtain predicted values of y from the model fit in step 1.\\
\item
  Calculate residual values using the predicted y values.\\
\item
  Calculate bisquare weights for each residual value using the following
  formula:\\
  \[max(1 - (residual/(6*mar))^2, 0)^2\]
\item
  Assign weights to each data point according to its residual value. If
  the residual value is positive, assign the bisquare downweight. If the
  residual is zero or negative, leave the weight at 1.\\
\item
  Repeat steps 1-5 with updated weights at each iteration for \(k\)
  iterations, with 20 iterations as the default.\\
\item
  After \(k\) iterations of updating the weight vector, fit a LOESS
  model (span = 1) and obtained predicted and residual values.
\end{enumerate}

The subsequent prediction methods for shoulder location are based on the
residuals calculated from the fit to the global structure of each land.
One method uses penalized two-class classification techniques to
classify each data point into ``LEA'' or ``GEA'', while the second uses
Bayesian changepoint analysis to predict the data points at which the
shoulders begin on either side.

\subsection{Two-Class Classification}

\subsection{Bayesian Changepoint Analysis}

The idea behind the changepoint approach is that within either the left
GEA, right GEA, or the LEA, the global structure is consistent and can
either be described by a line with zero slope, a line with positive
slope for the right GEA, or a line with negative slope for the left GEA.
Finding the points where the GEAs and LEA meet is treated as a problem
of model selection. That is, the best fitting statistical model, in
terms of the magnitude of the likelihood, should be the one which
assumes that the points at which the global structure changes align with
where the GEAs and LEA meet. These points of global structural change
are what we will call changepoints. Thus, our model will be defined in a
piecewise fashion. In practice there are also complex additional
patterns which may exist for a number of reasons, but this large scale
structural assumption remains generally reasonable. The complex smaller
scale patterns can be thought of as the dependence in the data after
accounting for the global structure. Because of the nature of the model
which we consider, it becomes necessary for computational reasons to
perform a couple of additional data preprocessing steps. Specifically,
we will scale the residuals from the robust LOESS procedure, and we will
impute missing values. In the next section, we describe the model that
we will use to identify changepoints, after which we will describe the
estimation procedure which we use. Details of the additional data
preprocessing steps can be found in the appendix.

\subsubsection{Bayesian Model Formulation}

Before introducing the model, we introduce some notation. First, let
\(\{Y(x_i): i = 1,2, ..., n\}\) denote the set of random variables
representing the residuals from the robust LOESS procedure at the values
\(x_i\). For simplicity, also assume that \(x_1 < x_2 < ... < x_n\).
Also, let \(c_l\) be the value of the left changepoint and \(c_r\) be
the value of the right changepoint. Here, the left changepoint is where
the left GEA meets the LEA, and the right changepoint is where the right
GEA meets the LEA. Also, denote the median centered \(x\) values as
\(x'_i = x_i - \tilde{x}\) where \(\tilde{x}\) is the median \(x\)
value. As mentioned in the previous paragraph, the complex small scale
patterns, such as the striae, will be modeled through a covariance
structure on the data that will be allowed to differ between each GEA
and between the GEAs and LEA. We will construct the covariance matrices
from the exponential covariance function
\(K(x, x';\sigma, \ell) = \sigma^2 e^{-\frac{|x - x'|}{\ell}} = cov(Y(x), Y(x'))\).
The differences in covariance matrices for the GEAs and LEA will be
reflected in the parameters \(\sigma\) and \(\ell\). The data model that
we consider is then,

\begin{align}
(Y(x_1), Y(x_2), ..., Y(x_{k_1})) &\sim N(\beta_{01}\mathbbm{1} + \beta_{11} x'_{1:k_1}, \Sigma_1(\sigma_1, \ell_1)) \\
(Y(x_{k_1 + 1}), Y(x_{k_1 + 2}), ..., Y(x_{k_2})) &\sim N(0, \Sigma_2(\sigma_2, \ell_2)) \\ 
(Y(x_{k_2 + 1}), Y(x_{k_2 + 2}), ..., Y(x_n)) &\sim N(\beta_{02}\mathbbm{1} + \beta_{12} x'_{k_2 + 1:n}, \Sigma_3(\sigma_3, \ell_3)),
\end{align}

where \(x_{k_1} < c_l \leq x_{k_1 + 1}\) and
\(x_{k_2} < c_r \leq x_{k_2 + 1}\) Here, \(x_{1:k}\) denotes the column
vector \((x_1, x_2, ..., x_k)^\top\), and \(\mathbbm{1}\) denotes the
vector of ones. Indpendence is assumed between each of these three
distributions for simplicity.

Thus the parameters that need to be estimated include the four mean
parameters in the GEAs, the six covariance parameters (two for each of
the three areas), and the two changepoint parameters, \(c_l\) and
\(c_r\).

The above model encapsulates the essence of the approach. However, there
are a few difficulties. The first difficulty is that there are not
always two GEAs in a particular land. There may be one GEA, or the land
may only consist of the LEA. Thus, the above model is actually
conditional on there being two GEAs in the data. We also define models
for when there is one GEA on the left, one GEA on the right, or no GEAs.
The models are defined in an essentially identical way. Conditional on
there being only one GEA, the left GEA model is defined as,

\begin{align}
(Y(x_1), Y(x_2), ..., Y(x_{k})) &\sim N(\beta_{0}\mathbbm{1} + \beta_{1} x'_{1:k}, \Sigma_1(\sigma_1, \ell_1)) \\
(Y(x_{k + 1}), Y(x_{k + 2}), ..., Y(x_{n})) &\sim N(0, \Sigma_2(\sigma_2, \ell_2)),
\end{align}

and the right GEA model is defined as,

\begin{align}
(Y(x_{1}), Y(x_{2}), ..., Y(x_{k})) &\sim N(0, \Sigma_1(\sigma_1, \ell_1)) \\ 
(Y(x_{k + 1}), Y(x_{k + 2}), ..., Y(x_n)) &\sim N(\beta_{0}\mathbbm{1} + \beta_{1} x'_{k + 1:n} \Sigma_2(\sigma_2, \ell_2)).
\end{align}

Finally, conditional on there being no GEAs in the data, the model is
simply

\begin{align}
(Y(x_{1}), Y(x_{2}), ..., Y(x_{n})) &\sim N(0, \Sigma(\sigma, \ell)).
\end{align}

Thus, estimating the changepoint locations also involves selecting the
most appropriate model. In order to avoid confusion, we have slightly
abused notation and, for example, \(\Sigma_1(\sigma_1, \ell_1)\) as it
is estimated in the two changepoint model is \emph{not} the same as
\(\Sigma_1(\sigma_1, \ell_1)\) from either of the one changepoint
models, and \(\Sigma_1(\sigma_1, \ell_1)\) is also \emph{not} the same
between the two one changepoint models. As another example, \(\beta_0\)
is \emph{not} the same between each of the one changepoint models. So,
to be clear, duplication of notation in \emph{different} models is not
meant to imply that those parameters are shared between models.

Ultimately, these above four models are each individually fitted, and
each model above is given a prior. From there, we do model selection in
the formal Bayesian way, selecting the most probable model.
Simultaneously with selecting the most probable model, we also use the
maximum a posteriori estimator for the changepoint locations.

In order to complete a Bayesian model specification, we need priors on
each of the parameters in each model as well as each model. We will
assume independence between each parameter a priori. For each length
scale \(\ell\), we will assume \(\ell \sim \text{Gamma}(3,5)\). For each
standard deviation, we will assume
\(\sigma \sim \text{Half-Normal}^{+}(0,1)\), where
\(\text{Half-Normal}^{+}(\cdot,\cdot)\) is notation for the normal
distribution restricted to the positive real numbers. For intercept
parameters, \(\beta_{01}, \beta_{02}, \beta_0 \sim N(0, 10)\). For the
slope parameters, the preceding trend deviates slightly. For any slope
that corresponds to the \emph{left} GEA, \(\beta_1\) or \(\beta_{01}\),
we will assume that the slope can not be positive. That is,
\(\beta_1, \beta_{01} \sim \text{Half-Normal}^{-}(0,10)\), where
\(\text{Half-Normal}^{-}(\cdot, \cdot)\) is notation for the normal
distribution restricted to the negative real numbers. Contrastingly, for
any slope that corresponds to the \emph{right} GEA, \(\beta_1\) or
\(\beta_{02}\), we will assume that the slope can not be negative. That
is, \(\beta_1, \beta_{01} \sim \text{Half-Normal}^{+}(0,10)\). For the
changepoint locations, we assume a uniform prior
\(\pi(c_l, c_r) \propto I(a < c_l < c_r - \gamma < b - \gamma)\). Here,
\(a\) and \(b\) are some values close to the edges of the data. How
close those values are to the edges is a parameter that is set manually.
Further, we include another hyperparameter, \(\gamma\), which can be set
so that the changepoints are not allowed to be too close to each other.
This is also a parameter that is set manually. Lastly, we assume a
uniform prior over all four models.

\section{Results}

\section{Conclusions}

\section{References}

\bibliographystyle{jfs-authoryear}
\bibliography{bibliography}

\end{document}