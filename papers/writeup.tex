% !TeX program = pdfLaTeX
\documentclass[12pt]{article}
\usepackage{amsmath}
\usepackage{amsthm}
\usepackage{graphicx,psfrag,epsf}
\usepackage{enumerate}
%\usepackage[numbers]{natbib}
\usepackage[nomarkers]{endfloat}
\usepackage{natbib}
\setcitestyle{numbers} 
\makeatletter % Reference list option change
\renewcommand\@biblabel[1]{#1. } % from [1] to 1
\makeatother %

\usepackage{booktabs}
\usepackage{longtable}
\usepackage{array}
\usepackage{adjustbox}
\usepackage{multirow}
\usepackage{subfig}
\usepackage[table,xcdraw]{xcolor}
\usepackage{wrapfig}
\usepackage{float}
%\usepackage{colortbl}
\usepackage[colorlinks]{hyperref}
\hypersetup{
  colorlinks=true,
  citecolor=black,
  linkcolor=black,
  urlcolor=blue}
  
\usepackage{pdflscape}
\usepackage{tabu}
\usepackage{threeparttable}
\usepackage{url} % not crucial - just used below for the URL

\usepackage{etoolbox}% http://ctan.org/pkg/etoolbox
\makeatletter
\patchcmd{\subsection}{\bfseries}{\relax}{}{}% Non-bold \subsection
\patchcmd{\subsubsection}{\bfseries}{\relax}{}{}% Non-bold \subsection
\makeatother


%\pdfminorversion=4
% NOTE: To produce blinded version, replace "0" with "1" below.
\newcommand{\blind}{0}

% DON'T change margins - should be 1 inch all around.
\addtolength{\oddsidemargin}{-.5in}%
\addtolength{\evensidemargin}{-.5in}%
\addtolength{\textwidth}{1in}%
\addtolength{\textheight}{1.3in}%
\addtolength{\topmargin}{-.8in}%

\newenvironment{definition}[1]% environment name 
{% begin code 
  \par\vspace{.75\baselineskip}\noindent 
  \textbf{Definition (#1)}\begin{itshape}% 
  \par\vspace{.5\baselineskip}\noindent\ignorespaces 
}% 
{% end code 
  \end{itshape}\ignorespacesafterend 
}

\providecommand{\tightlist}{%
  \setlength{\itemsep}{0pt}\setlength{\parskip}{0pt}}

\begin{document}

\def\spacingset#1{\renewcommand{\baselinestretch}%
{#1}\small\normalsize} \spacingset{1}



%%%%%%%%%%%%%%%%%%%%%%%%%%%%%%%%%%%%%%%%%%%%%%%%%%%%%%%%%%%%%%%%%%%%%%%%%%%%%%

\if0\blind
{
  \title{\bf Automated Groove Identification in 3D Bullet Land Scans (we'll change
the title)}

  \author{
        Kiegan Rice \thanks{The authors gratefully acknowledge \ldots{}} \\
    Department of Statistics, Iowa State University\\
     and \\     Nate Garton \\
    Department of Statistics, Iowa State University\\
     and \\     Ulrike Genschel \\
    Department of Statistics and CSAFE, Iowa State University\\
     and \\     Heike Hofmann \\
    Department of Statistics and CSAFE, Iowa State University\\
      }
  \maketitle
} \fi

\if1\blind
{
  \bigskip
  \bigskip
  \bigskip
  \begin{center}
    {\LARGE\bf Automated Groove Identification in 3D Bullet Land Scans (we'll change
the title)}
  \end{center}
  \medskip
} \fi

\bigskip
\begin{abstract}

\end{abstract}

\noindent%
{\it Keywords:} 3 to 6 keywords, that do not appear in the title
\vfill

\newpage
\spacingset{1.45} % DON'T change the spacing!

\section{Background}

Thanks to \cite{Hamby}, we can do stuff. \citet{Hamby}

\section{Data Source}

\section{Methodology}

We first need to remove the global structure of the bullet land.

\subsection{Global Structure Removal}

The non-traditional data structure necessitates employing
non-traditional methods to model and remove the global structure. The
data are made up of two competing structures: the LEA data, of which we
would like to model the global structure, and the GEA data, which we
would like to consider as outlying data. Traditional statistical
modeling techniques minimize the least squared vertical distance from
each data point to a fit line; this results in undue influence by GEA
points, which pull any fit lines towards their unusual points.

While bullets are traditionally circular, it is unwise to use a rigidly
quadratic model to fit the global structure. We cannot assume that fired
bullets will retain a neatly circular shape, especially at the level of
detail scans are captured. The significant amount of physical pressure
that acts upon bullets as they are fired through a barrel also can lead
to some warping or slight deformations \emph{(find a citation from JFS
or AFTE about warping/deformation of bullets?)}. Finally, the placement
of the land relative to the plane of reference when a 3D scan is being
captured can vary slightly, meaning that the 2D crosscuts can be
slightly tilted or rotated. This will not translate into a clean
quadratic-shaped(?) crosscut.

To avoid the potential risks arising from using a quadratic linear
model, we instead use a locally weighted regression (LOESS) which fits
linear regression models on small pieces of the data and combines
predictions to result in a non-parametric predicted fit of the data
structure.

However, since LOESS is still rooted in traditional regression
techniques, it is unable to adequately identify and address the
separation between GEA and LEA data structures. To address this, we
implement a robust version of LOESS which iteratively downweights
unusual data points and re-fits a LOESS model to each land. This robust
LOESS is an adapted version of the robust LOESS proposed by
\cite{Cleveland1}.

This model is fit as follows:

The subsequent prediction methods for shoulder location are based on the
residuals calculated from the fit to the global structure of each land.
One method uses penalized two-class classification techniques to
classify each data point into ``LEA'' or ``GEA'', while the second uses
Bayesian changepoint analysis to predict the data points at which the
shoulders begin on either side.

\subsection{Two-Class Classification}

\subsection{Bayesian Changepoint Analysis}

\section{Results}

\section{Conclusions}

\section{References}

\bibliographystyle{jfs-authoryear}
\bibliography{bibliography}

\end{document}